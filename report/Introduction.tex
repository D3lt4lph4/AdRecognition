\section{Introduction}

This Machine Learning assignment aim to take a look at different machine learning techniques in order to diferantiate between advertissments and legitimate contents on web pages. The final use of this kind of classifier could be for instance for programs such as adblock which allow the blocking of advertissements within web pages.\\

This report will detail the process used to create models in order to differanciate between advertissments (ads) and non advertissments (non-ads) contents within internet web pages.\\

The three models used are svm, random Forest and Neural network since they all are perfectly fit for classification problems.

\begin{itemize}
  \item SVM : Support Vector Machines are machine learning algorithms that try to find an hyperplane that separate the data (each class on one side) while minimising the error on the misclassified points;
  \item Random Forest : They are the composition of multiple decision trees which are then use to predict the class of an input vector through the mean of each tree prediction;
  \item Neural Network : They are classifier which aim to imitate the way the brain works, they use layer made up of neurons to predict a vector's class.
\end{itemize}
\vspace{\baselineskip}

Each model will be tested following a series of testing methods in order to find the best model possible.
The testing methods will be carring out in the following order :
\begin{itemize}
  \item First approach, simple test of the methods;
  \item Parameter optimization, opimization of one of the parameter used;
  \item Priors \& Noramlization, test of the models taking into account repartion and normalization of the data;
  \item Cross validation, use of the kfold method to find the best parameter.
\end{itemize}
\vspace{\baselineskip}

The code given in the annexes is the final code with all the methods mixed together, nevertheless it is possible to have access to the previous version of the code which are branched on a git repository. All the models are saved so it is possible to calculate each of the outputs obtained after each step of the process. Finaly, the whole project will require/provide the following :
\begin{itemize}
  \item CMake 2.6 (at least) (r);
  \item Opencv 2.4.13 (r);
  \item Working executables for each classifier (p);
  \item A working executable to test the models (p).
\end{itemize}
