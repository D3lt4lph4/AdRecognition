\section{Parameter optimization}

We've seen that the results after the simple approach (random choice of the parameter [either usual value or complete random]) can give good result but can also give under expectation ones depending on the method used.

Thus we now try to optimize a parameter used for each methods in order to find the best value to use each time. The data used will be the training set split in two and randomised. The first new set will be the new training set and the second the validation one (for calculating the error). The following results are caluclated with the testing set.

\subsection{SVM}

On this subsection, the svms will be calculated using the function auto-train and a kfold segmentation of two (one is not accepted by the function). The dataset used is not the split one.
For the stopping point of the algorithm the max iteration and the epsilon accuracy will be used. They will vary depending on the kernel used.

\subsubsection{linear}

The first type of model used was the linear one. The stopping criteria was of either a 1000000 iterations (low value to keep consistancy with the following chapter which will require more calculations) or 0.001 for epsilon.

Results :
\begin{itemize}
  \item Correct classification : 90.65\%;
  \item Wrong classification : 9.35\%;
  \item False positive non ad : 9.07\%;
  \item False positive ad : 0.29\%.
\end{itemize}

The final selected parameter is the following : 0.5 for C. As we can see there is improvement within the results, we go from 81.87\% of correct classification to 90.65\%.

\subsubsection{Polynomial}

For this one, the number of itertion was lowered down to 100000 because the time required for the optimization was too long.

Results :
\begin{itemize}
  \item Correct classification : 94.62\%;
  \item Wrong classification : 5.38\%;
  \item False positive non ad : 4.53\%;
  \item False positive ad : 0.85\%.
\end{itemize}

The final parameters were the following ones :
\begin{itemize}
  \item degree : 343;
  \item gamma : 0.00001;
  \item coef0 : 19.6;
  \item C : 0.5.
\end{itemize}


\subsubsection{Radial Basis Function}

We kept 100000 iterations.

Results :
\begin{itemize}
  \item Correct classification : 91.78\%;
  \item Wrong classification : 8.22\%;
  \item False positive non ad : 7.65\%;
  \item False positive ad : 0.57\%.
\end{itemize}

The final parameters were the following ones :
\begin{itemize}
  \item gamma : 0.00015;
  \item C : 312.5.
\end{itemize}


\subsubsection{Sigmoid}

Still a 1000 iterations

Results :
\begin{itemize}
  \item Correct classification : 83.85\%;
  \item Wrong classification : 16.15\%;
  \item False positive non ad : 16.15\%;
  \item False positive ad : 0\%.
\end{itemize}

The final parameters were the following ones :
\begin{itemize}
  \item gamma : 0.00015;
  \item coef0 : 0.1;
  \item C : 0.5.
\end{itemize}

\subsection{Neural Network}

For the neural network, the layers used are as follow :
  \begin{itemize}
    \item input layer, 1558, the number of point in the vector inputs;
    \item output layer, 2 outputs the number of class;
    \item hidden layer, this is the parameter we seek to optimize;
  \end{itemize}

 For this model only the number of cell in the hidden layer will be optimized, not the number of hidden layer (doc here). And the data will be randomized before use. The result shown are with the best parameter each time.

  Results :
  \begin{itemize}
    \item Correct classification : 91.22\%;
    \item Wrong classification : 8.78\%;
    \item False positive non ad : 6.80\%;
    \item False positive ad : 1.983\%.
  \end{itemize}

  The best number of cells for the hidden layer was of 50;

  \begin{figure}[h]
   \centering
   \includegraphics[scale=0.5]{../images/NNPO.png}
   \caption{Class diagram}
  \end{figure}

\subsection{Random forest}
For the random forrest i only played with the following parameter as it's the same with more, only giving more calculations :
  \begin{itemize}
    \item number of variable randomly selected sqrt(n) (usually the best number);
  \end{itemize}
Results :
\begin{itemize}
  \item Correct classification : 91.22\%;
  \item Wrong classification : 8.78\%;
  \item False positive non ad : 8.22\%;
  \item False positive ad : 0.57\%.
\end{itemize}

\begin{figure}[h]
 \centering
 \includegraphics[scale=0.5]{..data/images/RFPO.png}
 \caption{Class diagram}
\end{figure}
