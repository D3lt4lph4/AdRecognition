\section{Data Description}

The project will be carried with the following set of data :

If one takes a closer look to the data itselfit can be found that each vector correspond to a few big number at the begining and a lot of zero and some small numbers among them. One thing to do as pretrement of the data could be to apply the PCA in order to reduce the dimension of the vectors. This would have many advantages, the models can be trained only on the data that carries most of the information, thus it allows to reduced the timeof training. On an other topic, memory is not a problem here due to the small size of the dataset, but storing a lot of zeros, is memory consuming for no reason which can saturate extremily quickly the RAM of a pc (plus storage on database takes a lot more space) thus reduction can help improve performance and memory consuption. But here since the PCA would most likely not impact on the classifiers results and since memory is not an issue, the PCA will not be performed even if it could be a good following to this assigment.

The data is divided in the following way, the original data is pre-cut into two subset one of 353 samples for testing across all the models and the other of 2006 samples for training all of the models. The class are spread over both of the subset to keep the initial ratio of the first dataset (0.161509).
