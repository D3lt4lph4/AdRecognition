\section{Conclusion}

Depending on the model we get various precisions, if we hypothesis that the final user will want a model that makes not mistake on the false positive for the ads the model we could select are the following ones :
\begin{itemize}
  \item svmLinear
  \item svm Polynomial
  \item svm sigmoid
  \item svm rbf
  \item Neural network
  \item Random forest
\end{itemize}

So far we can categorize the result in two classes, the one that allow a user to categorize the non-ad data with almost a 100\% on the results and the one that allow the user to categorize the ad with almost certainty but on the detriment of non add recognition. Depending on the aim of the user, choising one or the other is up to him.

We used different method to improve our results but on side that wasn't treated in this project was the size of the data and the time used to train the models. In order to reduce the size of the data (which is mainly composed of zero values) one could use the PCA. It would allow the user to reduce the size of the imput quite importantly and at the same time accelerate the training processus.
